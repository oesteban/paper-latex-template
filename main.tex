%!TeX TXS-program:compile = txs:///lualatex/[--shell-escape]
%!TeX TXS-program:compile = txs:///latexmk/[-shell-escape]
\PassOptionsToPackage{table,xcdraw}{xcolor}
\documentclass[9pt]{nmeth}
% Use the onehalfspacing option for 1.5 line spacing
% Use the doublespacing option for 2.0 line spacing
% Please note that these options may affect formatting.
% Additionally, the use of the \newcommand function should be limited.
\usepackage{ragged2e}

\usepackage{xifthen}

\usepackage[inline]{enumitem}
\newboolean{review}
\newboolean{linenum}

\setboolean{review}{true}
\setboolean{linenum}{true}

% Template feature toggles (default: minimal build with no external tools)
% Enable these when adding real content that needs each feature.
\newif\iftemplatebibliography
\templatebibliographyfalse
\newif\iftemplatesvg
\templatesvgfalse
\newif\iftemplateminted
\templatemintedfalse

% Optional custom font override (LuaLaTeX/XeLaTeX).
% Keep a single family in this example; uncomment and replace as needed.
% \ifdefined\setmainfont
%   \setmainfont{Your Font Family Name}
% \fi

\usepackage{lineno}
\usepackage[version=4]{mhchem}
\usepackage{siunitx}

\usepackage{array}
\newcolumntype{C}[1]{>{\let\newline\\\arraybackslash\hspace{0pt}}m{#1}}
\newcolumntype{M}[1]{>{\raggedright\let\newline\\\arraybackslash\hspace{0pt}}m{#1}}
\newcolumntype{N}[1]{>{\tiny\raggedright\let\newline\\\arraybackslash\hspace{0pt}}m{#1}}
\newcolumntype{L}[1]{>{\raggedright\arraybackslash}m{#1}}

\usepackage{multirow}
\usepackage{booktabs}
\usepackage{dcolumn}

\DeclareSIUnit\Molar{M}
\DeclareSIUnit\b{\second\per\milli\meter\squared}

\usepackage{xcolor}
\usepackage{graphicx}
\iftemplatesvg
  % Safe default for template mode: load svg package without Inkscape conversion.
  % Switch to `inkscape=true` if you need on-the-fly SVG conversion.
  \usepackage[inkscape=false,inkscapeversion=1]{svg}
\fi
\usepackage{soul}
\usepackage{tikz}
\usetikzlibrary{calc}
\usetikzlibrary{shapes.misc}
\usetikzlibrary{fit}

\iftemplateminted
  % Requires shell-escape and a working Pygments installation.
  \usepackage{minted}
  % \usemintedstyle{monokai}
  \setminted{bgcolor=white!5}
  % \captionsetup[table]{skip=0pt}
\fi

\usepackage{placeins}
\usepackage{lettrine}
\usepackage{xspace, setspace}

\ifthenelse{\boolean{review}}
{
  \setboolean{linenum}{true}
  \colorlet{revcolor}{blue!70}
  \usepackage[colorinlistoftodos,textsize=tiny, textwidth=4.2cm]{todonotes}
  \newcommand{\revcomment}[2][]{%
    \ifthenelse{\equal{#1}{}}{}{\todo{\setstretch{0.2} #1}}%
    \begingroup \color{revcolor} #2\endgroup%
  }

  \newcommand{\revinline}[2][]{%
    \ifthenelse{\equal{#1}{}}{}{\todo[inline, inlinewidth=0.7cm, noinlinepar]{#1}}%
    \begingroup \color{revcolor} #2\endgroup%
  }
  
  \renewcommand{\baselinestretch}{1.5} 
  \setstcolor{red}
  \let\cancel\st
}
{
  \definecolor{revcolor}{gray}{0.0}
  \newcommand{\todo}[2][]{}
  \newcommand{\revcomment}[2][]{#2}
  \newcommand{\revinline}[2][]{#2}
  \newcommand{\cancel}[2][]{}
  \renewcommand{\baselinestretch}{1.0}
}

\usepackage{marginnote}

\newcommand\wordcount{%
    \immediate\write18{texcount main.tex > 'count.txt'}%
(\verbatiminput{count.txt}words)}

% \newcommand\wordcount{%
%     \immediate\write18{texcount -sub=section main.tex  | grep "Section" | sed -e 's/+.*//' | sed -n \thesection p > 'count.txt'}%
% (\input{count.txt}words)}

\newcommand{\detailtexcount}[1]{%
  \immediate\write18{texcount -merge -sum -q #1.tex output.bbl > #1.wcdetail }%
  \verbatiminput{#1.wcdetail}%
}

\colorlet{verylightgray}{lightgray!15}

% Create an inline list environment
\newlist{inlinelist}{enumerate*}{1}
\setlist*[inlinelist,1]{%
  label=(\roman*),
}

% Make the font in the SI commands match the text font
\sisetup{detect-all}

% Define some colors
\definecolor{whitesmoke}{rgb}{0.96, 0.96, 0.96}

%%%%%%%%%%%%%%%%%%%%%%%%%%%%%%%%%%%%%%%%%%%%%%%%%%%%%%%%%%%%
%%% ARTICLE SETUP
%%%%%%%%%%%%%%%%%%%%%%%%%%%%%%%%%%%%%%%%%%%%%%%%%%%%%%%%%%%%
\title{%
\textit{Paper Title Placeholder}: Subtitle Placeholder
}

\author[1\space\faIcon{envelope}]{Maya Thompson}
\author[2]{Daniel Rivera}
\author[1]{Priya Nair}
\author[3]{Lucas Bennett}
\author[4]{Elena Kovacs}

\affil[1]{Department of Biomedical Engineering, North River Institute of Technology, Portland, OR, USA}
\affil[2]{Department of Biostatistics, Summit Coast University, Boston, MA, USA}
\affil[3]{Center for Brain Imaging, Lakeview Medical College, Chicago, IL, USA}
\affil[4]{Department of Psychiatry, Redwood State University School of Medicine, Seattle, WA, USA}

\contrib{\faIcon{envelope}\space Corresponding author}
\corr{maya.thompson@example.org}{\faIcon{envelope}}

% \contrib[\authfn{1}]{These authors contributed equally to this work}
% \contrib[\authfn{2}]{These authors also contributed equally to this work}
% \presentadd[\authfn{1}]{Department of Psychology, Stanford University, CA, USA}
\makeatletter
\newbox\mybox
\newcount\test
\test0
\newcommand{\defcoder}[3][]{%
    \tikzset{every coder/.style={text=magenta,color=black!20, fill opacity=0.3, #1}}%
}

\defcoder[]{}{}

\newcommand{\code@DoHighlight}[1]{
    \node[outer sep = -13pt, inner sep = 0pt,
        fit=(begin code) (end code),
        every coder, this coder,
        rounded rectangle,#1,
        fill
    ]{} ;
}

\newcommand{\code@BeginHighlight}[1]{
  \coordinate[yshift=-\dp\mybox-1pt,#1] (begin code) at (0,0) ;
}

\newcommand{\code@EndHighlight}[1]{
  \coordinate[yshift=\ht\mybox+1pt,#1 ] (end code) at (0,0) ;
}

\newdimen\code@previous
\newdimen\code@current

\DeclareRobustCommand*\code[2][]{%
  \setbox\mybox\hbox{#2}
  \tikzset{this coder/.style={#1}}%
  \SOUL@setup
  %
  \gdef\ht@possiblenoleftround{}
  \def\SOUL@preamble{%
    \begin{tikzpicture}[overlay, remember picture]
      \code@BeginHighlight{xshift=3pt}
      \code@EndHighlight{xshift=-3pt}
    \end{tikzpicture}%
  }%
  %
  \def\SOUL@postamble{%
    \begin{tikzpicture}[overlay, remember picture]
      \code@EndHighlight{xshift=-3pt}
      \expandafter\code@DoHighlight\expandafter{\ht@possiblenoleftround}
    \end{tikzpicture}%
  }%
  %
  \def\SOUL@everyhyphen{%
    \discretionary{%
      \SOUL@setkern\SOUL@hyphkern
      \SOUL@sethyphenchar
      \tikz[overlay, remember picture] \code@EndHighlight{} ;%
    }{%
    }{%
      \SOUL@setkern\SOUL@charkern
    }%
  }%
  %
  \def\SOUL@everyexhyphen##1{%
    \SOUL@setkern\SOUL@hyphkern
    \hbox{##1}%
    \discretionary{%
      \tikz[overlay, remember picture] \code@EndHighlight{} ;%
    }{%
    }{%
      \SOUL@setkern\SOUL@charkern
    }%
  }%
  %
  \def\SOUL@everysyllable{%
    \begin{tikzpicture}[overlay, remember picture]
      \path let \p0 = (begin code), \p1 = (0,0) in \pgfextra
        \global\code@previous=\y0
        \global\code@current =\y1
      \endpgfextra (0,0) ;
      \ifdim\code@current < \code@previous
        \expandafter\code@DoHighlight\expandafter{\ht@possiblenoleftround,rounded rectangle right arc=none}
        \gdef\ht@possiblenoleftround{rounded rectangle left arc=none}
        \code@BeginHighlight{}
      \fi
    \end{tikzpicture}%
    \ttfamily\color{magenta}\the\SOUL@syllable
    \tikz[overlay, remember picture] \code@EndHighlight{} ;%
  }%
  \SOUL@{#2}
}
\makeatother

\makeatletter
\renewcommand*\refstepcounter[1]{\stepcounter{#1}%
\protected@edef\@currentlabel
{\csname p@#1\expandafter\endcsname\csname the#1\endcsname}}
%\newcounter{supplbox}
%\def\supplboxautorefname{Suppl. Box}

\runningtitle{ANALYSIS}

\newcommand{\discidx}{{\fontfamily{qcr}\selectfont
Discr%
}}
\makeatother

\newcommand{\eg}{e.g.,\xspace}
\newcommand{\etc}{etc.\xspace}
\newcommand{\ie}{i.e.,\xspace}

\newcommand{\invivo}{\textit{in vivo}\xspace}

\newcommand{\bspline}{B-Spline\xspace}

\newcommand{\bvals}{$b$-values\xspace}
\newcommand{\bvecs}{$b$-vectors\xspace}
\newcommand{\qspace}{$q$-space\xspace}

\newcommand{\threedvolreg}{\textit{3dVolreg}\xspace}
\newcommand{\afni}{\textit{AFNI}\xspace}
\newcommand{\ants}{\textit{ANTs}}
\newcommand{\antsreg}{\textit{antsRegistration}\xspace}
\newcommand{\datalad}{\textit{DataLad}\xspace}
\newcommand{\dipy}{\textit{DIPY}}
\newcommand{\dockerhub}{\textit{Docker Hub}\xspace}
\newcommand{\eddy}{\textit{eddy}\xspace}
\newcommand{\fmriprep}{\textit{fMRIPrep}\xspace}
\newcommand{\fsl}{\textit{FSL}\xspace}
\newcommand{\git}{\textit{git}\xspace}
\newcommand{\gitannex}{\textit{git-annex}\xspace}
\newcommand{\githubpackreg}{\textit{GitHub Package registry}\xspace}
\newcommand{\mcflirt}{\textit{MCFLIRT}\xspace}
\newcommand{\nifreeze}{\textit{NiFreeze}\xspace}
\newcommand{\nufo}{NuFO\xspace}
\newcommand{\openneuro}{\textit{OpenNeuro}\xspace}
\newcommand{\qsiprep}{\textit{QSIPrep}\xspace}
\newcommand{\qsirecon}{\textit{QSIRecon}\xspace}
\newcommand{\scikitlearn}{\textit{scikit-learn}\xspace}
\newcommand{\shoreline}{\textit{SHORELine}\xspace}
\newcommand{\smacthree}{\textit{SMAC3}\xspace}

%%%%%%%%%%%%%%%%%%%%%%%%%%%%%%%%%%%%%%%%%%%%%%%%%%%%%%%%%%%%
%%% ARTICLE START
%%%%%%%%%%%%%%%%%%%%%%%%%%%%%%%%%%%%%%%%%%%%%%%%%%%%%%%%%%%%
\begin{document}
\maketitle
\begin{abstract}
% Nature Computational Science (Analysis): typically 100--150 words, no references.
This is a placeholder abstract.
Replace this text with a concise summary of the manuscript motivation, methods, main results, and conclusions.
\end{abstract}

\newpage

\ifthenelse{\boolean{linenum}}
{
  \linenumbers
}
{
}
\lettrine{T}\-his is placeholder introductory text for an Analysis manuscript.
Replace with an unheaded introduction that defines the question, establishes context,
and states the main contributions.

This section should remain as plain paragraphs with no ``Introduction'' heading,
following Nature Computational Science Analysis format.

\section{Results}
\label{sec:results}

\subsection{Primary finding placeholder}
\label{subsec:results_primary}

Placeholder text describing the primary result of the analysis.
Replace with concise, evidence-driven statements that reference figures/tables.

\begin{figure}[h!]
  \centering
  \fbox{\rule{0pt}{2.5in}\rule{0.95\textwidth}{0pt}}
  \caption[Figure 1 placeholder]{%
  \textbf{Figure 1 | Placeholder title.}
  Replace with the main result figure and a complete legend.
  }
  \label{fig:result1}
\end{figure}

\subsection{Comparative analysis placeholder}
\label{subsec:results_comparative}

Placeholder text for secondary or comparative analyses.
Replace with additional quantitative findings and robustness checks.

\begin{table}[h!]
\caption{Table 1 | Placeholder summary of key metrics.}
\label{tab:result1}
\small
\begin{tabular}{p{0.45\textwidth} p{0.2\textwidth} p{0.2\textwidth}}
\toprule[\cmidrulewidth]
\textbf{Metric} & \textbf{Condition A} & \textbf{Condition B} \\
\toprule[\cmidrulewidth]
Placeholder metric 1 & -- & -- \\
Placeholder metric 2 & -- & -- \\
\bottomrule[\cmidrulewidth]
\end{tabular}
\end{table}

\section{Discussion}
\label{sec:discussion}

Placeholder discussion text.
Replace with interpretation of findings, implications, limitations, and future directions.
For Analysis manuscripts, keep Discussion as continuous prose without subheadings.

\section{Methods}
\label{sec:methods}

\subsection{Study design and data sources}
\label{subsec:methods_design}

Placeholder text for data sources, inclusion criteria, preprocessing inputs, and study design.

\subsection{Data processing workflow}
\label{subsec:methods_processing}

Placeholder text for computational workflow, model setup, and quality control.

\subsection{Statistical analysis}
\label{subsec:methods_statistics}

Placeholder text for statistical models, uncertainty quantification,
multiple-testing control, and software versions.

\clearpage

\section*{Data availability}
\label{sec:data_availability}

Placeholder text for data accessibility and repository identifiers.

\section*{Code availability}
\label{sec:code_availability}

Placeholder text for code repository URLs, version tags, and licenses.

\section*{Acknowledgements}
\label{sec:acknowledgements}

Placeholder text for funding, institutional support, and acknowledgements.

\section*{Author contributions}
\label{sec:author_contributions}

Conceptualization: placeholder.
Methodology: placeholder.
Software: placeholder.
Investigation: placeholder.
Writing -- original draft: placeholder.
Writing -- review \& editing: placeholder.
Supervision: placeholder.

\subsection*{Diversity, Equity, and Inclusion statement}
\label{sec:dei_statement}

We are committed to fostering diversity, equity, and inclusion throughout this work.
Study design, collaboration practices, and dissemination plans were developed to reduce barriers to participation and to support equitable access to methods and results.
Future iterations of this project will continue to evaluate representation, accessibility, and inclusion across all project stages.

\section*{Competing interests}
\label{sec:competing_interests}

The authors declare no competing interests.
%TC:endignore

\iftemplatebibliography
\bibliography{references}
\fi

\end{document}
